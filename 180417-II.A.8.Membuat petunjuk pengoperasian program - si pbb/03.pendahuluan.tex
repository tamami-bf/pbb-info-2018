\chapter{PENDAHULUAN}

Dalam era keterbukaan informasi, maka segala sesuatunya dapat langsung diakses oleh masyarakat demi tercapainya sikap saling percaya antara masyarakat dan pemerintah. Terlebih masyarakat wajib pajak.

Sudah sejak lama tertanam pada benak masyarakat wajib pajak bahwa biaya yang dikeluarkan oleh wajib pajak tidak selalu sampai ke Kas Daerah, ada kalanya dana tersebut digunakan oleh oknum yang tidak bertanggung jawab bahkan untuk dikembalikan ke Kas Daerah sebagaimana mestinya memakan waktu cukup lama.

Untuk meminimalisir hal tersebut di atas, adalah dengan membangun sistem pembayaran yang langsung terhubung ke Bank Tempat Pembayaran.

Dengan dibangunnya sistem informasi ini, masyarakat wajib pajak dapat mengetahui apakah dana yang telah disetorkan kepada petugas pemungut pajak PBB-P2 telah disetorkan ke Bank Kas Daerah, hal ini menjadikan masyarakat wajib pajak mampu melakukan kontrol terhadap aliran dana pajaknya.

Selain itu, karena datanya ditampilkan untuk seluruh tahun pajak yang pernah diterbitkan Surat Pemberitahuan Pajak Terhutang (SPPT)-nya, maka masyarakat wajib pajak pun menjadi tahu dan dapat melakukan verifikasi data pembayaran, di tahun pajak mana saja dia telah membayar, dan tahun pajak mana saja yang belum dibayarkan, atau mungkin melakukan konfirmasi pembayaran terkhusus untuk tahun pajak 2013 dan sebelumnya dimana pencatatan pembayaran atas PBB-P2 masih belum terkelola dengan baik.