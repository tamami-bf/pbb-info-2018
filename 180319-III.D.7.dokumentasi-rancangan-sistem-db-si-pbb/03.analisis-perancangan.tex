\chapter{ANALISIS DAN PERANCANGAN SISTEM BASIS DATA}

\section{Fungsi atau Kegunaan Sistem Basis Data}

Sistem basis data pada sistem informasi pembayaran Pajak Bumi dan Bangunan sektor Perdesaan dan Perkotaan adalah sebagai sumber utama data pencatatan pembayaran, karena kondisi pencatatan pembayaran bagi tiap objek pajak tersimpan dalam sistem basis data untuk sistem informasi atau aplikasi SISMIOP (Sistem Manajemen Informasi Objek Pajak) untuk pajak bumi dan bangunan sektor perdesaan dan perkotaan, maka fungsi sistem basis data ini bagi aplikasi atau sistem informasi pembayaran pajak bumi dan bangunan sektor perdesaan dan perkotaan ini hanya sebagai tempat untuk melakukan akses saja.

\section{Struktur Data}

Struktur data untuk tiap tabel yang digunakan pada sistem informasi atau aplikasi pembayaran pajak bumi dan bangunan sektor perdesaan dan perkotaan di Kabupaten Brebes ini adalah seperti berikut :

\subsection{Tabel SPPT}

Tabel ini selain mencatatkan ketetapan untuk tiap objek pada tiap tahun pajak, tabel ini juga mencatatkan status pembayarannya apakah sudah lunas atau belum. Struktur tabelnya adalah seperti pada gambar \ref{fig:struktur-tabel-sppt} berikut ini :

\begin{figure}[H]
	\centering
	\includegraphics[width=0.5\textwidth]{./resources/struktur-tabel-sppt}
	\label{fig:struktur-tabel-sppt}
	\caption{Struktur Tabel \texttt{SPPT}}
\end{figure}

Pengambilan informasi pada tabel \texttt{SPPT} ini hanya beberapa \textit{field} atau kolom saja, yaitu :

\begin{itemize}
	\item Nomor Objek Pajak, yang terdiri dari \textit{field} atau kolom \texttt{kd\_propinsi}, \texttt{kd\_dati2}, \texttt{kd\_kecamatan}, \texttt{kd\_kelurahan}, \texttt{kd\_blok}, \texttt{no\_urut}, dan \texttt{kd\_jns\_op}.
	\item Tahun pajak pada \textit{field} atau kolom \texttt{thn\_pajak\_sppt}
	\item Nama wajib pajak pada \textit{field} atau kolom \texttt{nm\_wp\_sppt}
	\item Besarnya pajak terhutang pada \textit{field} atau kolom \texttt{pbb\_yg\_harus\_dibayar\_sppt}
	\item Status pembayaran pada \textit{field} atau kolom \texttt{status\_pembayaran\_sppt}
\end{itemize}

\subsection{Tabel DAT\_OBJEK\_PAJAK}

Tabel \texttt{DAT\_OBJEK\_PAJAK}, digunakan untuk menampilkan informasi mengenai objek pajak seperti alamat, luas bumi dan bangunan, serta Nilai Jual Objek Bumi dan Bangunan. Struktur tabel dari \texttt{DAT\_OBJEK\_PAJAK} adalah seperti pada gambar \ref{fig:struktur-dat-op} berikut ini :

\begin{figure}[H]
	\centering
	\includegraphics[width=0.5\textwidth]{./resources/struktur-tabel-dat-op}
	\caption{Struktur Tabel \texttt{DAT\_OBJEK\_PAJAK}}
	\label{fig:struktur-dat-op}
\end{figure}

Pengambilan informasi pada tabel \texttt{DAT\_OBJEK\_PAJAK} ini pada beberapa \textit{field} atau kolom seperti berikut :

\begin{itemize}
	\item Alamat, akan menggunakan gabungan dari \textit{field} atau kolom \texttt{jalan\_op}, \texttt{blok\_kav\_no\_op}, \texttt{rw\_op}, dan \texttt{rt\_op}.
	\item Luas bumi akan menggunakan \textit{field} atau tabel \texttt{total\_luas\_bumi}.
	\item Luas bangunan akan menggunakan \textit{field} atau tabel \texttt{total\_luas\_bng}.
	\item Nilai Jual Objek Pajak (NJOP) bumi akan menggunakan \textit{field} atau tabel \texttt{njop\_bumi}.
	\item Nilai Jual Objek Pajak (NJOP) bangunan akan menggunakan \textit{field} atau tabel \texttt{njop\_bng}.
\end{itemize}

\subsection{Tabel DAT\_SUBJEK\_PAJAK}

Tabel \texttt{DAT\_SUBJEK\_PAJAK} ini digunakan untuk menampilkan informasi mengenai subjek pajaknya seperti nama dan alamatnya. Struktur tabel dari \texttt{DAT\_SUBJEK\_PAJAK} ini adalah seperti pada gambar \ref{fig:struktur-dat-sp} berikut ini :

\begin{figure}[H]
	\centering
	\includegraphics[width=0.5\textwidth]{./resources/struktur-tabel-dat-sp}
	\caption{Struktur Tabel \texttt{DAT\_SUBJEK\_PAJAK}}
	\label{fig:struktur-dat-sp}
\end{figure}

Informasi pada tabel \texttt{DAT\_SUBJEK\_PAJAK} yang digunakan ada pada beberapa \textit{field} atau kolom berikut :

\begin{itemize}
	\item Nama subjek pajak pada \textit{field} atau kolom \texttt{nm\_wp}
	\item Alamat subjek pajak pada \textit{field} atau kolom \texttt{jalan\_wp}, \texttt{blok\_kav\_no\_wp}, \texttt{rw\_wp}, \texttt{rt\_wp}, \texttt{kelurahan\_wp}, dan \texttt{kota\_wp}.
\end{itemize}

\subsection{Tabel REF\_KECAMATAN}

Untuk tabel \texttt{REF\_KECAMATAN} digunakan hanya untuk menampilkan informasi nama Kecamatan dimana objek berada. Struktur tabel untuk \texttt{REF\_KECAMATAN} ini seperti terlihat pada gambar \ref{fig:struktur-ref-kec} berikut ini :

\begin{figure}[H]
	\centering
	\includegraphics[width=0.5\textwidth]{./resources/struktur-tabel-ref-kec}
	\caption{Struktur Tabel \texttt{REF\_KECAMATAN}}
	\label{fig:struktur-ref-kec}	
\end{figure}

Informasi yang digunakan pada tabel \texttt{REF\_KECAMATAN} ini ada pada beberapa \textit{field} atau kolom berikut ini :

\begin{itemize}
	\item Nomor Identifikasi Kecamatan, pada \textit{field} atau kolom \texttt{kd\_propinsi}, \texttt{kd\_dati2}, dan \texttt{kd\_kecamatan}.
	\item Nama Kecamatan, pada \textit{field} atau kolom \texttt{nm\_kecamatan}
\end{itemize}

\subsection{Tabel REF\_KELURAHAN}

Tabel \texttt{REF\_KELURAHAN} pun digunakan hanya untuk menampilkan nama Kelurahan / Desa dimana objek pajak berada. Struktur tabel \texttt{REF\_KELURAHAN} ini seperti terlihat pada gambar \ref{fig:struktur-ref-kel} berikut ini :

\begin{figure}[H]
	\centering
	\includegraphics[width=0.5\textwidth]{./resources/struktur-tabel-ref-kel}
	\caption{Struktur Tabel \texttt{REF\_KELURAHAN}}
	\label{fig:struktur-ref-kel}
\end{figure}

Informasi pada tabel \texttt{REF\_KELURAHAN} yang digunakan ada pada beberapa \textit{field} atau kolom berikut ini :

\begin{itemize}
	\item Nomor Identifikasi Kelurahan / Desa pada \textit{field} atau kolom \texttt{kd\_propinsi}, \texttt{kd\_dati2}, \texttt{kd\_kecamatan}, dan \texttt{kd\_kelurahan}.
	\item Nama Desa / Kelurahan pada \textit{field} atau kolom \texttt{nm\_kelurahan}
\end{itemize}

\section{Hubungan Antar Entitas}

Dari tabel-tabel yang terbentuk pada bagian sebelumnya, dalam sistem ini akan membentuk sebuah jaringan relasi antar tabel dengan bentuk seperti pada gambar \ref{fig:db-diagram} berikut ini :

\begin{figure}[H]
	\centering
	\includegraphics[width=0.5\textwidth]{./resources/db-diagram}
	\caption{Diagram Relasi \textit{Entity}}
	\label{fig:db-diagram}
\end{figure}

Titik utama akses aplikasi ini ada pada tabel \texttt{SPPT}, dimana nantinya tiap data pada tabel ini akan memiliki relasi n:1 dengan tabel \texttt{DAT\_OBJEK\_PAJAK}, ini karena tiap objek pajak yang tercatat akan memiliki banyak data SPPT untuk tiap tahun pajak.

Setiap data pada tabel \texttt{DAT\_OBJEK\_PAJAK} akan memiliki relasi 1:1 dengan data pada tabel \texttt{DAT\_SUBJEK\_PAJAK}. 

Sedangkan hubungan atau relasi antara tabel \texttt{DAT\_OBJEK\_PAJAK} dengan \texttt{REF\_KECAMATAN} dan \texttt{DAT\_OBJEK\_PAJAK} dengan \texttt{REF\_KELURAHAN} adalah n:1, dimana tiap 1 (satu) data pada tabel \texttt{REF\_KECAMATAN} atau \texttt{REF\_KELURAHAN} akan memiliki banyak objek pada tabel \texttt{DAT\_OBJEK\_PAJAK}.